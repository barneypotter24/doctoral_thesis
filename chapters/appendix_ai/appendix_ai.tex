% !TeX root = ../../thesis.tex
\chapter{GenAI use statement}
\label{appendix:ai}

\section*{Generative AI during the PhD period}

\textit{(The following text was---of course---written using generative AI tools.)}

Since 2020 (when my PhD began), generative AI has undergone a remarkable transformation from a promising but niche technology to a mainstream force reshaping numerous industries.
The release of GPT-3 by OpenAI in June 2020 marked a significant turning point, demonstrating unprecedented capabilities in natural language processing and generation.
This was followed by breakthrough image generation models like DALL-E, Midjourney, and Stable Diffusion, which captured public imagination by creating sophisticated artwork from text descriptions.
The accessibility of these tools through user-friendly interfaces helped democratize AI technology, allowing artists, writers, developers, and everyday users to experiment with AI-powered creation.

The launch of ChatGPT in late 2022 represented another watershed moment, introducing conversational AI to a massive global audience and sparking intense interest from businesses, educators, and the general public.
This was quickly followed by competing models from major technology companies and a wave of AI-powered applications across sectors like healthcare, education, and business.
The rapid advancement of these tools has sparked important discussions about AI safety, ethics, and regulation, while also driving significant investment and research in the field.

Generative AI can serve as a valuable research and learning aid in academic writing when used appropriately and ethically.
Students and researchers can leverage these tools to brainstorm ideas, generate outlines, find relevant sources to investigate, debug code, or get feedback on draft arguments.
However, it's crucial to maintain academic integrity by always acknowledging any AI assistance in the research or writing process, verifying all AI-generated information against reliable sources, and never presenting AI-generated text as one's own work.
Many institutions have developed specific policies around AI use in academic settings that typically emphasize using these tools to enhance rather than replace critical thinking and original analysis.
The key is to treat generative AI as a collaborative tool that supports the research and writing process while ensuring that the final work represents the author's own understanding, analysis, and intellectual contribution to the field.

\textit{(From here onward in this appendix everything is written without the use of GenAI, except where it is explicitly quoted.)}

\section*{KU Leuven's policy (abridged)}
KU Leuven created and published a university-wide policy on the use of generative AI.
This policy is summarized at \url{https://www.kuleuven.be/english/education/student/educational-tools/generative-artificial-intelligence} (published 15 February 2024; last updated 25 October 2024).

In their policy, they state:

\begin{quote}
    Ku Leuven opts to \textbf{engage Generative Artificial Intelligence (GenAI) in the long term}. After all, GenAI tools have found their way into the professional field, where you will encounter them when you graduate. It's important for students to understand how GenAI works so you can maintain academic standards and retain ownership of the written text.
\end{quote}

KU Leuven's policy further highlights the importance of \textit{transparency} in how GenAI was used for any assignment or document (a policy with which I fully agree).
They do state `` `Being transparent' is essential to ensure that the teaching staff can always assess your knowledge, understanding and/or skills correctly \ldots A good way to be transparent is \textbf{to keep track of how and why you used GenAI} \ldots [for example w]rite down why GenAI was used: to save time, to combat writer's block, to stimulate reflection, to manage increasing stress, to better understand a concept, to translate, to experiment with GenAI, etc.''

In order to be completely transparent about my own use of GenAI in the writing of this thesis, I believe it is most appropriate to describe how I have used GenAI, explicity mention where I have \textit{not} used GenAI, and to provide a brief example of how exactly I have used the tools available to me.

\section*{My use of GenAI}

\subsection*{How and where I have used GenAI}
For the writing of my thesis, I chose to make use of Claude AI 3.5 ``Sonnet'' (\url{https://claude.ai}) to assist me in writing for my general introduction and conclusion chapters.
I selected Claude AI as my GenAI tool based on the recommendation of Jordan Harrod (\url{https://www.jordanharrod.com/}), a PhD student at the Harvard-MIT Health Sciences and Technology program, and prominent scientific communicator on AI technologies.
Their recommendations for how to choose the correct AI, as well as my personal preference toward the ``tone'' of Claude AI compared to other prominent tools led me to my choice.

I have personally struggled with writers block, as well as with the process of drafting text---I have a much easier time editing/rewriting than I do coming up with sentences and paragraphs to begin with.
To address this, I chose to provide ClaudeAI with prompts for the content of sentences and paragraphs I was struggling to draft.
I would then \textit{personally edit the GenAI-drafted text}; I would frequently reword, rewrite, or even completely remove text written by the tool.
Additionally, I made sure to hand-curate any information that had been drafted to ensure factual correctness, attribution where it was relevant, and overall tone were eactly as I would like them to be.

\subsection*{Where I have not used GenAI}
I have only used GenAI tools to assist with the general introduction and conclusion chapters of this thesis.
All of the text in the main content chapters (Chapter~\ref{ch:chapter1}, Chapter~\ref{ch:chapter2}, and Chapter~\ref{ch:chapter3}) was written without the use of GenAI.
As these chapters represent the primary intellectual contribution that I have made during my PhD I have chosen to have them be entirely written in my own (and in the case of manuscripts co-authors') words.

\subsection*{An example}

Below is an example taken from the top of my conversatioin with ClaudeAI, as well as an example from further into the thread which caused the tool to draft proportionally more text.
I have copied these logs directly from the ClaudeAI interface, and the only modifications I've made are to get rid of artifacts of copy/pasting \LaTeX code that was input into ClaudeAI.

\vspace{2em}
\hrule width \hsize \kern 1mm \hrule width \hsize

\textbf{Barney Isaksen Potter (BIP):} As a Doctoral Dissertation Assistant, your job is to assist PhD students in drafting doctoral dissertation.

   You specialize in using Bayesian phylogenetic methods to answer questions in molecular epidemiology.
   
   Your work has focused on SARS-CoV-2 and hepatitis B virus, however the skills that you have are broadly applicable to all viral systems.

\textbf{ClaudeAI (CAI):} Understood. As a Doctoral Dissertation Assistant specializing in Bayesian phylogenetic methods for molecular epidemiology, I'm here to help PhD students draft their dissertations, particularly those working on viral systems like SARS-CoV-2 and hepatitis B virus. My expertise can be applied to various aspects of doctoral research in this field, including:

   1. Developing research questions and hypotheses
   2. Designing studies using Bayesian phylogenetic approaches
   3. Analyzing genomic data of viruses
   4. Interpreting results in the context of molecular epidemiology
   5. Structuring and writing dissertation chapters
   6. Creating figures and tables to represent phylogenetic data
   7. Explaining complex Bayesian methods in accessible language

    While my focus has been on SARS-CoV-2 and hepatitis B virus, I can assist with applying these methods to other viral systems as well. How can I help you with your doctoral dissertation today? Do you have any specific questions or areas where you need assistance?

\textbf{BIP:} Your job is to assist with the drafting of the first chapter of the dissertation, entitled "General Introduction." This chapter will serve to orient the readers with the background information that will be necessary to understand the analyses and results that will be given in the remaining chapters. 

    The chapter (and therefore snippets that I provide) are written in LaTeX. 
    You must give answers and suggestions in plain text.
    
    This chapter begins:
    """
    All living things exist in a constant state of conflict with the biological agents that exploit them in the arena of evolution.
    Since the advent of civilization humans have sought an upper-hand in this biological arms race: we harvest the flora and fauna that surround us through agriculture and domestication to our benefit, and we fiercely seek to defeat the plants, animals, and disease-causing pathogens that would do us harm.
    While we have largely found means to overcome the challenges of our natural environment to lengthen and improve our lives, the eradication of parasites, bacteria, and viruses remains one of the great challenges facing mankind.
    Indeed, as the human population grows and urbanizes we become increasingly vulnerable to new diseases more rapidly than ever before.
    Concurrent with the exponential growth and globalization of humans is a demand for increased quality of living; which in turn necessitates the human colonization of natural environments that were previously uninhabited by humans in search of land and resources.
    This intrusion into environments puts us into greater contact with animals that host diseases that our species has never before encountered, leading to an ever-increasing risk of emerging infectious diseases (\verb|\gls{eid}|s) jumping from their natural reservoir into human beings, and spreading at unprecedented speeds through their new susceptible hosts.
    Perhaps the greatest of these threats is that posed by viruses.\\
    
    Viruses are the root cause of some of the \verb|\gls{eid}|s that have caused the most human suffering and death over the last hundred years.
    They have been responsible for outbreaks that have affected local regions (e.g. Ebola virus in central and western Africa \verb|\&| MERS-CoV in the Arabian peninsula) to epidemics that can seasonally threaten large portions of the globe (e.g. Dengue virus, RSV, and West Nile Virus) to pandemics of global scale that threaten all people (e.g. seasonal influenza viruses, \verb|\gls{hiv}|, and \verb|\gls{sarscov2}|).
    In fact, viruses have been responsible for all major pandemics since the year 1900: six major influenza pandemics (ranging from the 1918 Spanish flu to the 2009 H1N1pdm09 ``swine flu''), the \verb|\gls{hiv}|/AIDS pandemic beginning in 1981, and the 2020 COVID-19 pandemic.
    Viruses also have some of the greatest future pandemic potential due to their ubiquity and fast-mutating nature.\\
    
    Fortunately, global public health efforts have made increasingly effective strides toward understanding and eliminating viral pathogens, perhaps most famously by eradicating smallpox through a global vaccination campaign ending in the 1980.
    A crucial step in the fight against viral pathogens has been the development of clinical and statistical methodologies that link the outcomes of viral epidemics (case counts, mortality rates, etc.) to the underlying determinants of pathogen spread that drive the epidemic---a field of study known as epidemiology.
    """
    
    Propose three options of two sentences each for how the final paragraph of this snippet can be concluded as a segue into the full General Introduction.

\textbf{CAI:} Here are three options for concluding the final paragraph of the snippet, each with two sentences that serve as a segue into the full General Introduction:

    Option 1:
    While traditional epidemiology has provided invaluable insights, the advent of molecular techniques has revolutionized our understanding of viral dynamics. This dissertation focuses on the application of Bayesian phylogenetic methods to molecular epidemiology, a powerful approach that combines genetic data with epidemiological information to unravel the complexities of viral transmission and evolution.
    
    Option 2:
    In recent years, the field of molecular epidemiology has emerged as a critical tool in our arsenal against viral pathogens, offering unprecedented insights into their spread and evolution. This dissertation explores the application of Bayesian phylogenetic methods to molecular epidemiological data, with a particular focus on SARS-CoV-2 and hepatitis B virus, demonstrating the broad applicability of these techniques in understanding and combating viral epidemics.
    
    Option 3:
    As our understanding of viral pathogens has grown, so too have the sophisticated tools we use to study them, with molecular epidemiology at the forefront of this scientific advancement. In this dissertation, we delve into the power of Bayesian phylogenetic methods in molecular epidemiology, showcasing how these approaches can be leveraged to gain deeper insights into viral transmission dynamics and evolutionary patterns across different pathogen systems.

\vdots

\textbf{BIP:} The next section is a brief introduction to the COVID-19 pandemic. It should include facts including: when and where it started, how quickly it moved around the globe, total case counts and deaths, and other relevant epidemiological information.

The section begins: "
Contemporary readers of this thesis will require no introduction to the COVID-19 pandemic, however..."

Finish drafting this section in three total pragraphs.

\textbf{CAI:} Here's a draft completion of the COVID-19 introduction section:

Contemporary readers of this thesis will require no introduction to the COVID-19 pandemic, however, it is essential to document its unprecedented scale and impact for historical context. The first reported cases of an unusual pneumonia emerged in Wuhan, China, in December 2019, with the causative agent identified as a novel coronavirus, SARS-CoV-2, in January 2020. Within three months, the virus had spread to nearly every country on Earth, leading the World Health Organization to declare a global pandemic on March 11, 2020, marking the beginning of what would become one of the most significant public health crises in modern history.

The rapid global spread of SARS-CoV-2 was facilitated by its high transmissibility, ability to spread through asymptomatic carriers, and the interconnected nature of our modern world. By the end of 2020, over 83 million cases and 1.8 million deaths had been reported globally. These numbers would continue to grow exponentially, with successive waves driven by new variants of concern, ultimately resulting in over 770 million reported cases and nearly 7 million reported deaths by the end of 2023—though excess mortality studies suggest the true death toll may be significantly higher. The pandemic's impact extended far beyond direct health outcomes, triggering unprecedented public health measures including widespread lockdowns, border closures, and mask mandates, while simultaneously catalyzing remarkable scientific achievements such as the rapid development of effective vaccines.

The COVID-19 pandemic also marked a watershed moment in the application of genomic surveillance and molecular epidemiology. The global scientific community's response was unprecedented in both scale and speed, with over 16 million SARS-CoV-2 genome sequences generated and shared publicly within the first three years of the pandemic. This massive dataset enabled real-time tracking of viral evolution and spread, including the identification and monitoring of variants of concern such as Alpha, Beta, Gamma, Delta, and Omicron. Each of these variants exhibited distinct transmission advantages and immune escape properties, fundamentally altering the trajectory of the pandemic and necessitating rapid adjustments to public health responses.
\hrule width \hsize \kern 1mm \hrule width \hsize
\vspace{2em}

For the first example, I liked the sound of the first sentence provided by Option 1, and combined it with a rewrite of the second sentence from Option 2.

In the second example, I first decided to wholesale remove the third paragraph, as I was not fond of the way that it sounded, and I knew that I would have a better opportunity to discuss some of the points that were suggested in later sections.
I then rewrote the second half of the first sentence and moved it to its own standalone paragraph.
Finally, I worked through the remaining two paragraphs to rewrite most sentences and add appropriate citations.

The final edited version contains some of the text provided by ClaudeAI, but is modified significantly:

Contemporary readers of this thesis will require no introduction to the COVID-19 pandemic, however, a quick overview will still be given for potential future readers.

The first reported cases of an atypical pneumonia was first noted in Wuhan, Hebei, China, in December 2019 \citep{zhou2020pneumonia}.
This outbreak of atypical pneumonia was promptly identified as a novel coronavirus \citep{wu2020new}
Within three months, the virus had globalized \cite{dong2020interactive}, leading the World Health Organization to declare a global pandemic on March 11, 2020 \cite{healthorganization2020coronavirus}.

Between 2020 and 2023, the COVID-19 pandemic demonstrated repeated resurgences, with successive waves driven by the emergence of novel variants.
This ultimately resulted in over 770 million reported cases and over 7\,000\,000 confirmed deaths \citep{ourworldindata-covid-explorer}.
The pandemic's impact extended far beyond direct health outcomes, resulting in unprecedented economic and social upheaval.
The COVID-19 pandemic also triggered public health measures across the globe such as  lockdowns, border closures, and mask mandates.
The pandemic officially ended in May 2023, however \gls{sarscov2} has now become endemic in the human population and continues to circulate.


%%%%%%%%%%%%%%%%%%%%%%%%%%%%%%%%%%%%%%%%%%%%%%%%%%
% Keep the following \cleardoublepage at the end of this file,
% otherwise \includeonly includes empty pages.
\cleardoublepage

% vim: tw=70 nocindent expandtab foldmethod=marker foldmarker={{{}{,}{}}}
